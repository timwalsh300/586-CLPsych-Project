% finalreport.tex - Final report for COSC 586

\documentclass{article}
\usepackage{times}
\usepackage{graphicx}
\usepackage{amsmath}
\usepackage{algorithm2e}
\usepackage[top=3cm, bottom=3cm, left=3cm, right=3cm]{geometry}

\begin{document}

\title{Predicting Change in a User's Mental State Based on Community Interaction in an Online Mental Health Support Forum}
\author{
Julien Han\\
Tim Walsh\\
COSC 586\\
Georgetown University\\
}
\maketitle

\begin{abstract}
This paper proposes a new task related to the 2016 and 2017 CLPsych shared task, and presents a system that attempts to address the new task. Rather than directly classifying forum posts according to a user's apparent state of distress as previously done, this task involves predicting how a user's mental state will change based on their interaction with the forum community. The system first leverages another system that was successful in the original task to label the entire ReachOut.com dataset, and then uses a machine learning approach to predict the change in a target user's mental state based on replies by other forum users. The goal is to find correlation between how other users interact with a target user, and how the target user's distress increases or decreases, allowing us to learn how a community can better help forum users in distress.
\end{abstract}

\section{Introduction and Motivation}

\paragraph{}In 2016 and 2017, CLPsych challenged the text mining community to assist moderators of the online mental health support forum, ReachOut.com. Moderators wanted a way to automatically classify new posts (by applying 'green,' 'amber,' 'red,' or 'crisis' labels) to quickly bring to their attention users in a state of distress who needed immediate support. CLPsych attracted 15 teams that submitted systems for this task, with top performing systems correctly labeling up to 95\% of the posts, and ReachOut.com now employs this technology on their forum\cite{milne}\cite{cohan2}.

\paragraph{}In trying to improve upon systems for this original task, based on previous research into peer influence and the 'copycat suicide' effect, we hypothesized that higher and lower risk users may form social clusters and thus influence each other in more negative or positive ways\cite{gladwell}\cite{phillips}\cite{stack}. Concretely, this could involve building a social link graph based on conversations on the forum, computing risk scores for each user in a manner similar to the PageRank algorithm, and then including the user's score as another feature in the representation of their posts.

\paragraph{}From that intuition arose questions about how to confirm or deny the hypothesis. Successfully using it to improve results in the original task would provide some degree of confirmation, but we reasoned that we could measure influence more directly by defining a new task. The new task involves predicting how a user's mental state will change based on their interaction with other members of the forum community.

\paragraph{}Additionally, we hope that if it is possible to predict changes in a user's mental state based on the activity of other forum users, then we can gain important insights from the trained model about how best help a forum user in distress. In particular, with a machine learning approach, extracting the most predictive features may teach us effective words and phrases to use, or topics to discuss, along with those we should avoid.

\section{Related Work}

\paragraph{}The 2016 CLPsych workshop proceedings present details and results from the work of 15 teams in the original task using the same ReachOut.com dataset. Every team used a machine learning approach with, at a minimum, bag-of-words features and a supervised learning method based the manually labeled subset of posts given for the task. Some teams incorporated a semi-supervised learning approach, leveraging datasets from Twitter and Reddit forums focused on depression. Teams selected many different combinations of features from the body text of each post as well as metadata associated with each post and user. Teams also used a variety of classifiers, with some in combination\cite{milne}. The top performing systems tended to focus on features from the body text, as opposed to metadata, such as token and character n-grams, part-of-speech tags, mood, sentiment, emotion, and topic. Top performing systems also used logistic regression and support vector machine classifiers. The highest performing system, by Kim et al, labeled the official test set with an F1 score of 0.42 and accuracy of 0.85\cite{kim}. Notably, some top performing systems included features from adjacent posts by other users. The intuition for this was to provide context, but it could also indicate influence.

\paragraph{}In follow-up work for the 2017 workshop, Cohan et al, improved their system to achieve an F1 score of 0.51 and accuracy of 0.95. Additionally, after labeling the entire dataset, Cohan et al used their results to explore other research questions. One discovery was that most users who were active for two or more months on the forum showed improvement in their apparent mental state as measured by the severity of their first and last posts: 81\% who began with a 'crisis' or 'red' post eventually posted content labeled 'amber' or 'green'\cite{cohan2}. Cohan et al note that this may be due to a positive influence by the forum community.

\section{Approach}

\paragraph{}We first restructure and filter the dataset to produce 'conversation sets.' A conversation set consists of an original post or posts by a target user $U$ on day 0, followed by $L$ posts by other users that mention $U$ by-name over some period of $N$ days. Note that this is distinct from the concept of a forum 'thread,' which has one original post and subsequent replies that generally relate to one topic but may not directly address the original post or user. We then create a feature vector representation of the $L$ posts in the set, along with the label for $U$'s initial state and some additional metadata for the set. Finally, our classifier aims to predict the label of $U$'s next post appearing after the $N$-day period. In order to determine $U$'s initial state and train our classifer, we leverage the 2017 Cohan et al system to obtain labels for all posts in the dataset.

\paragraph{}After constructing the sets, we filtered out all those sets where $U$ was a moderator. We did this because we are less interested in the mental state of moderators, and doing so helped to balance the training data since nearly all moderator posts are labeled green. We also removed block-quotes written by $U$, and all occurrences of $U$ in each set. These last two steps ensured that the classifier trained on the activity of other users instead of training on $U$ directly.

\paragraph{}We then extracted the following features for each set: TF-IDF weighted unigrams, bigrams, and trigrams (stop words removed using the NLTK list), the number of replies in the set, the number of replies by a moderator, the number of posts by $U$ during the $N$ day period, a simple average risk score for users replying to $U$, sentiment polarity, and sentiment subjectivity. We defined a risk score for a replying user as the percentage of their posts in the total dataset that were flagged. We determined polarity and subjectivity for all the replies lumped together using the TextBlob package. If a conversation set contained no replies, we set values for these last three features to be neutral, choosing 0.07 as a neutral risk score because 7\% of posts in the entire dataset are flagged.

\paragraph{}For the parameter $N$, we experimented with values ranging from 1 to 14. At very low values of $N$, it is obvious that sets will contain few replies. In fact 40\% of sets contain no replies at $N$ = 1, but we did not remove those sets because getting no replies could have predictive power. Only 13\% of sets were empty at $N$ = 14. For large values of $N$, however, we believe that we begin to lose relevance. For example, it seems unlikely that something said on the forum in January would affect $U$'s state in December. Similarly, we also had to place a limit on the timing of $U$'s 'next post,' and we chose $N$ + 7. That is, if we did not find a 'next post' to associate with a set within 7 days of the $N$-day period, we deleted that set. In addition to $N$-day periods, we also experimented with the idea of a 'conversation snapshot' which looked at two consecutive posts by $U$, agnostic of time, and the replies occurring between those posts, but results were not good.

\paragraph{}We tried three different classifiers. The first was a support vector machine with a linear kernel, the second was a ---, and third was ---. We implemented and trained all classifiers using the scikit-learn package.

\section{Experimental Setup}

\paragraph{}As we were defining this task ourselves, we had to reason about many details of the setup. The first detail was how many labels to use. To ensure the highest level of accuracy, we chose to use only 'green' and 'flagged,' where 'amber,' 'red,' and 'crisis' were lumped together as 'flagged,' because the Cohan et al system succeeded in this task with an F1 score 0.92. Another detail was determining the label for $U$'s initial state. If $U$ made just one post on day 0, we simply took the label of that post. In the case where $U$ made multiple posts on day 0, we tried three different methods. The 'absolute' method set the initial state to be flagged if any post on day 0 was flagged. The 'last post' method took the label of the final post made by $U$ on day 0. The 'majority' method set the initial state as flagged if 50\% or more posts on day 0 were flagged. The absolute method produced twice as many flagged cases and gave us the best initial results, so we performed all subsequent trials with this method. A third detail was determining how to define a 'reply' to include in a conversation set. We used several different types of replies. One is when $U$ is quoted with the forum's block-quote functionality. The others are the appearance of patterns such as 'Hey $U$, ...' or '...@$U$...' in the body text.

\paragraph{Dataset}The dataset provided by CLPsych for its 2016 shared task includes 65,755 posts from the ReachOut.com forum from July of 2012 to June of 2015. Labels for the entire dataset were provided by the 2017 Cohan et al system.

\paragraph{Evaluation Plan}We evaluated our system with 5-fold cross validation looking at F1 score, accuracy, precision, and recall. We separately evaluated conversation sets with a green initial state, sets with a flagged initial state, and all sets together (with initial state label as a feature).

\paragraph{Results}

\paragraph{Analysis}We achieved our highest F1 score of --- with the --- classifier using all features, for all conversation sets, at $N$ = 14.

\section{Conclusion}

\begin{thebibliography}{9}

	\bibitem{milne}
	David N. Milne, Glen Pink, Ben Hachey, and Rafael A. Calvo. 2016. Clpsych 2016 shared task: Triaging content in online peer-support forums. In Proceedings of the Third Workshop on Computational Lingusitics and Clinical Psychology. Association for Computational Linguistics, San Diego, CA, USA, pages 106–117.
	\bibitem{kim}
	Sunghwan Mac Kim, Yufei Wang, Stephen Wan, and Cecile Paris. 2016. Data61-CSIRO systems at the CLPsych 2016 Shared Task. In Proceedings of the 3rd Workshop on Computational Linguistics and Clinical Psychology: From Linguistic Signal to Clinical Reality, San Diego, California, USA, June 16.
        \bibitem{cohan2}
        Cohan, Arman, Sydney Young, Andrew Yates, and Nazli Goharian. "Triaging Content Severity in Online Mental Health Forums." arXiv preprint arXiv:1702.06875 (2017).
        \bibitem{gladwell}
        Gladwell, Malcolm. The Tipping Point: How Little Things Can Make a Big Difference. New York: Little, Brown and Company Hachette Book Group, 2002. eBook Edition.
        \bibitem{phillips}
        Phillips, David. The Influence of Suggestion on Suicide: Substantive and Theoretical Implications of the Werther Effect. American Sociological Review Vol. 39, No. 3 (Jun., 1974), pp. 340-354
        \bibitem{stack}
        Stack, Steven. "Media coverage as a risk factor in suicide." Journal of Epidemiology \& Community Health 57.4 (2003): 238-240.

\end{thebibliography}

\end{document}